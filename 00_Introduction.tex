\chapter{RTFM}

People who do technical support (and here I speak of myself)
are angry, bitter people. There are many
reasons for this, but they all boil down to one thing: the ``user'', who
they are supporting, is ignorant and content to remain so, interested
only in the quick fix and moving on with their life.

What we neglect to notice is that we are exactly the same way. We want
our toilet repaired, our brakes fixed, our taxes filed, and our hernia
cured, and really don't care much about the technical details involved.
We not plumbers, and have no desire to become so. The toilet is a tool
with which we perform a particular task, and we are interested in the
task, not in the tool.

So it is with software. The folks who use software, who we so
deprecatingly speak of as ``users'', have tasks they wish to perform.
They want to write a novel, balance a checkbook, draw a unicorn, or
shoot aliens. They don't care about configuration files and file paths.

Documentation is a necessary evil. Until we finally get around to
writing software that just works, all the time, and does what people
expect instead of what they say, we're going to need to write
documentation that explains away the foibles of imperfect products.

Documentation, of course, has just as many failings as the software
itself. It answers the wrong questions, often incorrectly, and often
does so unintelligibly. Then, once we're done writing them, we write
lengthy list of FAQs, which are the questions that we wish people would
ask so that we can give them esoteric answers giving deep insight into
the nature of our product, as well as showing off how much we know.

All of this results in a situation where when folks want to accomplish a
particular class, they are faced with a bewildering mish-mash of poorly
organized information that does not speak to their immediate needs, and
angry bitter self-titled experts telling them to just go Read The Fine
Manual.

The time has come to write a better FM. And the only way to do this is
to think of our customers as customers, not ``users'', and to understand
what their needs are. This can come only from listening to them, and not
presuming ahead of time that we know what all the right answers are, so
that we can work out ways to wedge their questions into those slots.

In this book, we try to give some tips on how to write a better manual,
starting with understanding who your audience is and crafting your
writing to their particular needs.

Now that I've gotten my clever introduction out of the way, let's get
started.
