\chapter{RTFM}

People who do technical support (and here I speak of myself)
are angry, bitter people. There are many
reasons for this, but they all boil down to one thing: the ``user'', who
they are supporting, is ignorant and content to remain so, interested
only in the quick fix and moving on with their life.

What we neglect to notice is that we are exactly the same way. We want
our toilet repaired, our brakes fixed, our taxes filed, and our hernia
cured, and really don't care much about the technical details involved.
We are not plumbers, and have no desire to become so. The toilet is a tool
with which we perform a particular task, and we are interested in the
task, not in the tool. Come to think of it, we're often not even that
interested in the task. We just want to get it done, so that we can move
on to things that we do care about.

So it is with software. The folks who use software, who we so
deprecatingly speak of as ``users'', have tasks they wish to perform.
They want to write a novel, balance a checkbook, draw a unicorn, or
shoot aliens. They don't care about configuration files and file paths.

Documentation is a necessary evil. Until we finally get around to
writing software that just works, all the time, and does what people
expect instead of what they say, we're going to need to write
documentation that explains away the foibles of imperfect products.

Documentation, of course, has just as many failings as the software
itself. It answers the wrong questions, often incorrectly, and often
does so unintelligibly. Then, once we're done writing them, we write
lengthy list of FAQs, which are the questions that we wish people would
ask so that we can give them esoteric answers giving deep insight into
the nature of our product, as well as showing off how much we know.

All of this results in a situation where trying to accomplish a task
turns into a frustrating quest after the wrong things. Rather than 
performing the task, the ``user'' finds themself learning about a tool.
Each problem is met with angry, bitter,
self-titled experts telling you to just go Read The Fine Manual. The FM
is a bewildering mish-mash of poorly organized information that does not
speak to their immediate needs, but further questions only increase the
venom. ``RTFM'' is often interpreted as ``I don't know, but I'm not
about to admit that to you.''

The time has come to write a better FM. And the only way to do this is
to think of our customers as customers, not ``users'', and to understand
what their needs are. This can come only from listening to them, and not
presuming ahead of time that we know what all the right answers are, so
that we can work out ways to wedge their questions into those slots.

In this book, we try to give some tips on how to write a better manual,
starting with understanding who your audience is and crafting your
writing to their particular needs.

Now that I've gotten my clever introduction out of the way, let's get
started.

\section{Technical mumbo-jumbo}

This book is written using LaTeX, which is a typesetting language of
sorts. There's a chapter later in the book about selecting a
documentation format.

The source for this book is hosted on GitHub. There's a chapter
somewhere in this book about revision control, and about developing
documentation in a public fashion. The decision to write a book in
public should be carefully considered, based on what you intend to do
with the final product. While I (obviously) recommend it, it's not for
everyone. You can get the source for this book on GitHub at
http://github.com/rbowen/Write-A-Better-FM but I detailed instructions
about how to turn that source into a useful document are outside of the
scope of this book. There's a chapter somewhere about defining your
scope, too.

